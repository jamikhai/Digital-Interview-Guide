\subsection*{From Harris \& Harris Textbook: ``Computer Architecture \& Digital Design''}

\subsubsection*{Combinational Logic}
\begin{enumerate}
    \item Draw a two-input XOR gate using only NAND gates. What is the minimum number of gates required? \\
    \item Design a circuit that determines whether a month has 31 days based on a 4-bit input.
    \item What is a tristate buffer? Explain its function and typical use cases.
    \item Why are NAND gates considered universal? Justify your answer.
    \item Why can a circuit's contamination delay be shorter than its propagation delay?
\end{enumerate}

\subsubsection*{Sequential Logic}
\begin{enumerate}
    \item Create an FSM that detects the input sequence 1010. \\
    \item Design a serial FSM that performs two’s complement bit-by-bit.
    \item What is the difference between a latch and a flip-flop? When should each be used?
    \item Design an FSM that functions as a 5-bit counter.
    \item Implement an edge detector that outputs HIGH on a rising edge (0 to 1 transition).
    \item What is pipelining, and why is it useful in digital systems?
    \item Define negative hold time in the context of flip-flops.
\end{enumerate}

\subsubsection*{Timing and Metastability}
\begin{enumerate}
    \item Explain timing constraints for logic between two registers.
    \item If a buffer is added to the clock input of the second flip-flop, how does that affect the setup time requirement?
    \item What is metastability, and how do synchronizers mitigate its effects?
\end{enumerate}

\subsubsection*{Verilog/SystemVerilog}
\begin{enumerate}
    \item What is the difference between blocking and non-blocking assignments in Verilog?
    \item When should you use \texttt{always\_comb} vs. \texttt{always\_ff}?
    \item How do you write a clean and efficient testbench in SystemVerilog?
    \item Compare \texttt{case} statements with \texttt{if-else} constructs in Verilog.
    \item How do you define parameterized modules in Verilog?
\end{enumerate}

\subsubsection*{Microarchitecture \& FPGA Design}
\begin{enumerate}
    \item What are pipeline hazards, and how can they be resolved?
    \item Why don't modern CPUs use extremely deep pipelines (e.g., 100 stages)?
    \item Compare cache organizations: direct-mapped, set-associative, and fully-associative.
    \item What are the key differences in design approach between FPGAs and ASICs?
    \item Discuss the trade-offs involved in implementing FSMs on an FPGA.
\end{enumerate}

\subsubsection*{Miscellaneous Digital Design Topics}
\begin{enumerate}
    \item Compare clock gating and power gating. When is each used?
    \item How does clock skew impact setup and hold timing?
    \item What is the difference between synchronous and asynchronous resets?
    \item How can multiplication be implemented efficiently in digital circuits?
    \item What is the maximum possible result from multiplying two N-bit numbers?
\end{enumerate}

\subsection*{Interview Questions I Was Asked (Various Companies)}

\subsection*{Signal Processing}
\begin{enumerate}
    \item How is an analog signal converted to digital? (Cover discretization and quantization)
    \item What are sampling and aliasing, both in the time and frequency domains?
    \item How many bits are needed to represent a signal with N distinct values? (e.g., $\log_2(N)$)
    \item Compare FIR and IIR filters: benefits, limitations, and their phase characteristics.
\end{enumerate}

\subsection*{Digital Design}
\begin{enumerate}
    \item How can you build an OR gate using only NAND gates?
    \item How do you construct a 2x1 multiplexer using basic gates?
    \item Write the truth tables for JK and D flip-flops.
    \item Design a D flip-flop using a JK flip-flop.
    \item Design a 2-bit counter using JK flip-flops (use LSB as the clock for MSB).
    \item What are setup time and hold time in sequential circuits?
    \item \textbf{Bonus:} How do you derive the transfer function of an IIR filter?
\end{enumerate}

\subsection*{Verilog}
\begin{enumerate}
    \item What is the difference between synchronous and asynchronous resets? Provide Verilog examples.
    \item How do you swap two variables in Verilog with and without a temporary register?
    \item Compare blocking and non-blocking assignments in Verilog.
    \item What distinguishes a latch from a flip-flop?
    \item Which types of Verilog code will infer a latch, and why?
        \begin{figure}[H]
            \centering
            \includegraphics[width=0.6\linewidth]{images/latch-inference.jpg}
            \caption{Latch Inference in Verilog.}
        \end{figure}
    \item Describe the differences between \texttt{case}, \texttt{casez}, and \texttt{casex}.
    \item Determine the output of the following \texttt{casez} and \texttt{casex} statements:
        \begin{figure}[H]
            \centering
            \includegraphics[width=0.6\linewidth]{images/case-statements.jpg}
            \caption{Verilog Case Statements.}
        \end{figure}
    \item What is a glitch in digital circuits?
    \item How would you eliminate the glitch shown in this waveform and code?
        \begin{figure}[H]
            \centering
            \includegraphics[width=0.6\linewidth]{images/glitch-waveform.png}
            \caption{Glitch Waveform Example.}
        \end{figure}
    \item Write Verilog for a circuit with three input gates, three D flip-flops, and a single output gate:
        \begin{figure}[H]
            \centering
            \includegraphics[width=0.6\linewidth]{images/circuit-diagram.jpg}
            \caption{Circuit Diagram for Verilog Module.}
        \end{figure}
    \item Design an FSM that detects the sequence “011” and write the Verilog code.
    \item What is the difference between inter-delay and intra-delay statements?
    \item Given the code and delays below, what are the resulting logic outputs?
        \begin{figure}[H]
            \centering
            \includegraphics[width=0.6\linewidth]{images/delay-statements.png}
            \caption{Verilog Delay Statements.}
        \end{figure}
\end{enumerate}

\subsection*{Digital/Physical}
\begin{enumerate}
    \item Identify and explain timing violations in the following waveform:
        \begin{figure}[H]
            \centering
            \includegraphics[width=0.6\linewidth]{images/timing-waveform.png}
            \caption{Timing Violation Waveform.}
        \end{figure}
    \item Using given setup times, calculate the minimum viable clock period:
        \begin{figure}[H]
            \centering
            \includegraphics[width=0.6\linewidth]{images/timing-calculation.jpg}
            \caption{Timing Calculation Example.}
        \end{figure}
    \item Describe the design handoff from Verilog to floorplanning (e.g., 45nm project).
    \item How were power (VDD) and ground (GND) nets routed in your project?
    \item Explain the process and goals of clock tree synthesis.
    \item What makes clock lines unique in digital layouts?
    \item What are the typical challenges during place and route/layout?
    \item What file types (e.g., SDC) were used in your flow?
    \item Which static timing analysis tool was used to close timing?
\end{enumerate}

\subsection*{Miscellaneous}
\begin{enumerate}
    \item Describe how an FPGA functions.
    \item What is a Look-Up Table (LUT) in the context of FPGAs?
    \item Recall key topics from computer architecture (e.g., jump, branch, C-level mappings).
    \item What is a Phase-Locked Loop (PLL)?
    \item Define clock jitter and its impact on performance.
    \item Review sampling and aliasing — what issues can arise?
    \item What are the Fourier and Laplace transforms used for?
    \item Walk through a typical ASIC design flow and project life cycle.
    \item What is the difference between DRC and LVS checks?
    \item What are DACs and ADCs, and how are they applied in systems?
\end{enumerate}
