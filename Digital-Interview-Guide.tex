% Digital/FPGA/ASIC Hardware Design Interview Guide

\documentclass[11pt]{article}
\usepackage{float}
\usepackage{graphicx}
\usepackage{amsmath}
\usepackage{listings}
\usepackage{xcolor}
\usepackage{cite}
\usepackage[margin=0.5in]{geometry}
\bibliographystyle{ieeetr}

\lstdefinestyle{verilog}{ backgroundcolor=\color{gray!5}, language=Verilog,
    basicstyle=\ttfamily\small, keywordstyle=\bfseries\color{blue},
    commentstyle=\itshape\color{gray}, stringstyle=\color{orange},
    showstringspaces=false, numbers=left, numberstyle=\tiny, frame=single,
    breaklines=true }


\begin{document}

%=============================
% Title and Introduction
%=============================
\title{Digital/FPGA/ASIC Hardware Design Interview Guide}
\author{Justin Mikhail\\University of Alberta, 2025\\BSc. Computer Engineering, Spec. in Nanoscale System Design}
\date{}
\maketitle

\section*{Purpose}
This guide has been created to help you prepare for digital hardware interviews
(ASIC/FPGA) at a level expected from a recent graduate of a Computer
Engineering undergraduate program. It covers combinational/sequential logic,
HDL (Verilog/SystemVerilog), microarchitecture, timing analysis, and more. This
resource is purely for educational and learning purposes. Some questions
adapted from Harris \& Harris \cite{Harris2021}.

%=============================
% Preparation Insights
%=============================
\section{Preparation Insights}
\begin{itemize}
    \item \textbf{Know Your Resume:} Be prepared to discuss everything
    listed—projects, tools, and responsibilities—at both high and detailed
    levels, including drawing a block diagram of your project while talking
    through it.

    \item \textbf{Cross-Functional Readiness:} Expect comprehensive questions
    that may span verification, design, and software/firmware interaction.
    Interviewers often assess system-level understanding beyond RTL and
    understanding of good design practices.

    \item \textbf{Problem Solving Focus:} Interviewers value your approach and
    reasoning more than just getting the right answer. Communicate your thought
    process clearly and consider trade-offs. Practice explaining your thought
    process during problems, and don't be afraid to ask for a moment to think
    or ask to clarify assumptions. This demonstrates critical thinking and
    ensures you fully understand the problem before solving it.

    \item \textbf{Master the Fundamentals:} Ensure a strong understanding of
    core digital design concepts such as pipelining, FSMs, combinational and
    sequential logic, and memory addressing. Often an interviewer will ask you
    to explain a concept before you do a design exercise with it.

    \item \textbf{PPA and Timing} Be ready to discuss Power-Performance-Area
    (PPA) tradeoffs, Clock Domain Crossing (CDC), timing, and low-power
    strategies. Be well-versed in setup and hold times, clock skew, on-chip
    variations, power consumption, metastability, and timing paths. Once again,
    relate thought processes and design choices to Power, Performance, and
    Area.

    \item \textbf{RTL Design Topics:} Practice implementing sequence detectors,
    understand the difference between Mealy and Moore machines, and be able to
    write clear, synthesizable RTL. Know that an RTL testbench is just as
    important as the module itself. Understanding the simulation tool and
    reading waveforms is essential.

    \item \textbf{Leverage Your Experience:} Be able to explain your past
    roles, design challenges, and key decisions. Reflect on what you learned
    and how you grew from those experiences. Show humility and a willingness to
    learn from mistakes. Do your best to be charismatic and personable, as
    interviewers are also evaluating if they want to work with you.

    \item \textbf{Don't Rely on AI:} Study for interviews as you would for an
    exam, as you will not have access to AI tools during the interview. While
    AI can be a helpful tool for learning, it should not replace your own
    understanding and problem-solving skills.
\end{itemize}

%=============================
% Digital Design Fundamentals
%=============================
\section{Digital Design Fundamentals}

\subsection{Combinational Logic}
\begin{enumerate}
    \item What is DeMorgan's theorem?
    \item How can you build an OR gate using only NAND gates? Show the Boolean
    algebra and the circuit diagram.
    \item Draw a two-input XOR gate using only NAND gates. What is the minimum
    number of gates required?
    \item What is the difference between a combinational and a sequential
    circuit?
    \item What is a tristate buffer? How and why is it used?
    \item Are NAND gates universal? Why?
    \item Design a circuit that determines whether a month has 31 days based on
    a 4-bit input representing the month (1-12). Use a truth table and K-map.
    \item What is the boolean equation of a multiplexer?
    \item Explain the concept of fan-out in combinational circuits. How does it
    impact circuit design?
    \item What is the significance of propagation delay in combinational
    circuits? How can it be minimized?
    \item Design a priority encoder for a 4-bit input. Explain its
    functionality.
\end{enumerate}

\subsection{Sequential Logic}
\begin{enumerate}
    \item What is setup time and hold time?
    \item What determines the maximum clock frequency? How could you change a
    circuit if the clock frequency needed to be increased?
    \item Explain the difference between a Mealy and a Moore state machine.
    Which has more states? Which may be more prone to glitches?
    \item What is pipelining, and why is it useful in digital systems?
    \item Design a D flip-flop using a JK flip-flop.
    \item Design a 2-bit counter using JK flip-flops.
    \item Create an FSM that detects the input sequence \texttt{10X1}. (X =
    Don't Care)
    \item Design an edge detector circuit that outputs HIGH on a rising edge (0
    to 1 transition).
    \item What is metastability in flip-flops, and how can it be mitigated?
    \item Explain the concept of clock domain crossing and why it is important.
    What are 3 common methods to handle it?
    \item Draw a state transition diagram for a \texttt{10110} sequence detector, with
    and without detection overlap.
\end{enumerate}

%=============================
% Verilog / SystemVerilog
%=============================
\section{Verilog / SystemVerilog}

\subsection{Language Basics}
\begin{enumerate}
    \item How does Verilog differ from VHDL?
    \item What is the difference between synchronous and asynchronous resets?
    Provide Verilog examples and discuss their advantages and disadvantages in
    terms of synthesis and timing.
    \item When should \texttt{always\_comb} be used instead of
    \texttt{always\_ff} in SystemVerilog?
    \item Compare blocking (\verb|=|) and non-blocking (\verb|<=|) assignments
    in Verilog. When should each be used?
    \item Design an FSM that detects the sequence “011” and write the Verilog
    code. Draw the state (transition) diagram before writing any code. Bonus:
    Compare both Mealy and Moore module implementations.
    \item How do you swap two variables in Verilog with and without a temporary
    register? Provide a Verilog example for each.
    \item How do you define parameterized modules in Verilog?
    \item What is the purpose of the \texttt{generate} statement in
    SystemVerilog?
    \item How does \texttt{if-else} differ from \texttt{case} statements in SystemVerilog? When
    would you use one over the other? How may they differ in terms of
    synthesis?
    \item What is the purpose of the \texttt{initial} block in Verilog? How
    does it differ from the \texttt{always} block?
    \item Describe the differences between \texttt{case}, \texttt{casez}, and
    \texttt{casex}.
    \item Below are two Verilog snippets. Identify the issues in each causing a
    latch to be inferred.
    \lstinputlisting[style=verilog]{examples/latch_inference.v}

    \item Given the code and delays below, what are the logic values of the
    signals at t = 5?
    \lstinputlisting[style=verilog]{examples/delay_statements_tb.v}


    \item What is a glitch in digital circuits, and how would you eliminate the
    glitch shown in this waveform by changing the code?
    \lstinputlisting[style=verilog]{examples/glitch_code.v}
    \lstinputlisting[style=verilog]{examples/glitch_code_tb.v}

    \begin{figure}[H]
        \centering
        \includegraphics[width=0.7\linewidth]{images/glitch-waveform-drawn.png}
        \caption{Glitch Waveform Example.}
        \label{fig:glitch_waveform}
    \end{figure}

    \item Write a Verilog module for the following circuit given in the diagram
    by first writing a module for a D flip-flop and using 3 instances of it in
    the design. Bonus: Write a testbench for the module.
    \begin{figure}[H]
        \centering
        \includegraphics[width=0.6\linewidth]{images/circuit-diagram-drawio.jpg}
        \caption{Circuit Diagram for Verilog Module.}
        \label{fig:verilog_circuit}
    \end{figure}

    \item Write a Verilog module to infer a Block RAM (BRAM) in synthesis with
    a read and a write (R/W) port at different R/W clock frequencies. Include a
    testbench that demonstrates reading and writing to the BRAM. Bonus:
    Parameterize the BRAM depth and width.

    \item Create a 4-bit ripple-carry full adder in Verilog. Write a testbench that
    tests the adder with various inputs and checks the outputs. Include
    assertions to verify correctness. Bonus: Use a \texttt{generate} statement to
    instantiate a full adder for each bit.
\end{enumerate}

\subsection{Testbenches \& Verification}
\begin{enumerate}
    \item What is the goal of a good testbench in RTL?
    \item What is the impact of delay statements on simulation vs.\ synthesis?
    \item What is the difference between inter-delay and intra-delay statements
    in Verilog?
    \item How do you write a testbench for a module that includes clock and
    reset signals? Provide a simple example.
    \item What is the purpose of assertions in SystemVerilog, and how do you
    use them in testbenches?
    \item For a given Verilog exercise in the previous section, write a
    comprehensive testbench with a clock, reset, and randomized stimulus.
    Verify the output using assertions and visual inspection of the waveform.
\end{enumerate}

%=============================
% Signal Processing \& DSP
%=============================
\section{Signal Processing \& DSP}
\begin{enumerate}
    \item What is sampling in the context of digital signal processing?
    \item What is the Nyquist theorem, and how does it relate to sampling?
    \item How is an analog signal converted to digital?
    \item Explain the affects of aliasing in both time and frequency domains.
    \item What is a Fourier Transform, and how is it used in signal processing?
    Explain the meaning of a Fourier Transform in both time and frequency
    domains.
    \item What is a Fast Fourier Transform (FFT), and how does it differ from a
    Discrete Fourier Transform (DFT)?
    \item How many bits are needed to represent a signal with \(N\) distinct
    values?
    \item Compare FIR and IIR filters. What are the benefits, limitations, and
    phase characteristics?
    \item How do you derive the transfer function of an IIR filter given a
    schematic/diagram?
    \item What is the difference between a low-pass and high-pass filter?
\end{enumerate}

%=============================
% Timing \& Metastability
%=============================
\section{Timing}
\begin{enumerate}
    \item Explain why a circuit's contamination delay might be less than its
    propagation delay.
    \item Explain timing constraints for logic between two registers.
    \item What are the equations for setup and hold time? Explain the variables
    and meanings.
    \item If a buffer is added to the clock input of the second flip-flop, how
    does that affect setup time requirements?
    \item List a few ways to fix a setup time violation.
    \item List a few ways to fix a hold time violation.
    \item What is metastability, and how do synchronizers mitigate its effects?
    \item What is clock skew, and how does it impact setup and hold timing?
    \item Define clock jitter and its impact on performance.
    \item What is a Phase-Locked Loop (PLL)?
    \item Identify and explain potential timing violations in the following
    waveform.
    \begin{figure}[H]
        \centering
        \includegraphics[width=0.6\linewidth]{images/timing-waveform.png}
        \caption{Timing Waveform.}
        \label{fig:timing_violation}
    \end{figure}
    \item Using the given timing values, calculate the minimum viable clock
    period.
    \begin{figure}[H]
        \centering
        \includegraphics[width=0.6\linewidth]{images/timing-calculation.jpg}
        \caption{Timing Calculation Example.}
        \label{fig:timing_calc}
    \end{figure}
\end{enumerate}

%=============================
% Physical Design \& FPGA/ASIC Flow
%=============================
\section{Physical Design \& FPGA/ASIC Flow}
\begin{enumerate}
    \item Describe the difference between synthesis in FPGA vs ASIC design.
    What is the meaning of a netlist?
    \item How did you simulate your design? What tools did you use?
    \item How may power (VDD) and ground (GND) nets be typically routed on a
    chip?
    \item Explain the process and goals of clock tree synthesis. What
    considerations should be made for clock routing?
    \item What are the some challenges during place and route/layout?
    \item What is timing closure? Have you used a static timing analysis tool?
    \item Describe how an FPGA functions, and specifically, what is a Look-Up
    Table (LUT)?
    \item What is a constraints file in the context of FPGA design?
    \item Walk through a typical ASIC design flow and project life cycle.
    \item What is the difference between DRC and LVS checks? Why are each important?
    \item What are the advantages and disadvantages of using low-voltage
    transistors (LVT) in ASIC design? Relate your answer to PPA (Power,
    Performance, Area) trade-offs.
    \item What are the effects of changing drive strength of a transistor in an
    ASIC design? How does it affect PPA?
    \item What are static and dynamic power? How can you reduce each in an ASIC
    design? What is the equation for dynamic power?
\end{enumerate}

%=============================
% Miscellaneous \& Advanced Topics
%=============================
\section{Architecture \& Miscellaneous}
\begin{enumerate}
    \item What is an ALU?
    \item What is a stack data structure? What is queue data structure?
    \item What is a FIFO buffer? What happens if the write frequency is greater
    than the read frequency? How is the depth/size determined?
    \item What is heap memory, and how does it differ from stack memory?
    \item What is a smart pointer in C++? How does it differ from a pointer in
    C?
    \item How is C code eventually ran by a CPU? Describe the process from C
    code to machine code.
    \item What is a program counter, and how does it work in a CPU? How do jump
    or branch instructions affect it?
    \item How do you reverse a linked list in C?
    \item What is spatial and temporal locality in the context of cache memory?
    \item Compare clock gating and power gating. When is each used?
    \item How can multiplication be implemented efficiently in digital
    circuits?
    \item What is the maximum possible result from multiplying two (N)-bit
    unsigned numbers? How does this differ for signed numbers?
    \item Compare cache organizations: direct-mapped, set-associative, and
    fully-associative.
    \item What are pipeline hazards, and how can they be resolved? Why don't
    modern CPUs use extremely deep pipelines (e.g., 100 stages)?
    \item Discuss the trade-offs involved in different implementations of FSM
    encodings on an FPGA such as one-hot, binary, and gray code.
    \item What is a JTAG interface, and how is it used in hardware debugging?
    \item What are DACs and ADCs, and how are they applied in systems?
    \item "What is the resolution of a DAC or ADC, and how does it relate to
    bit depth and voltage levels?
    \item What is SPI and what signals are needed to implement it?
    \item What is I2C and what signals are needed to implement it?
    \item What is UART and what signals are needed to implement it?
\end{enumerate}

%=============================
% 9. \textit{“From Harris \& Harris Textbook”} (Optional Section)
%=============================
% \section{From Harris \& Harris, \emph{Computer Architecture \& Digital
% Design} textbook} \noindent\textit{(These questions are adapted from end of
% chapter "Interview Questions" of Harris \& Harris.) \cite{Harris2021}}

% \begin{enumerate} \item \textbf{Combinational Logic} \begin{enumerate} \item
%     Draw a two-input XOR gate using only NAND gates. What is the minimum
%     number of gates required? \item Design a circuit that determines whether
%     a month has 31 days based on a 4-bit input. \item What is a tristate
%     buffer? Explain its function and typical use cases. \item Why are NAND
%     gates considered universal? Justify your answer. \item Why can a
%     circuit's contamination delay be shorter than its propagation delay?
%     \end{enumerate} \item \textbf{Sequential Logic} \begin{enumerate} \item
%     Create an FSM that detects the input sequence \texttt{1010}. \item What
%     is the difference between a latch and a flip-flop? When should each be
%     used? \item Design an FSM that functions as a 5-bit counter. \item
%     Implement an edge detector that outputs HIGH on a rising edge (0 to 1).
%     \item What is pipelining, and why is it useful in digital systems? \item
%     Define negative hold time in the context of flip-flops. \end{enumerate}
%     \item \textbf{Timing \& Metastability} \begin{enumerate} \item Explain
%     timing constraints for logic between two registers. \item If a buffer is
%     added to the clock input of the second flip-flop, how does that affect
%     setup requirements? \item What is metastability, and how do synchronizers
%     mitigate its effects? \end{enumerate} \item
%     \textbf{Verilog/SystemVerilog} \begin{enumerate} \item What is the
%     difference between blocking and non-blocking assignments? \item When
%     should you use \texttt{always\_comb} vs.\
%         \texttt{always\_ff}? \item How do you write a clean and efficient
%         testbench in SystemVerilog? \item Compare \texttt{case} statements
%         with \texttt{if-else} constructs. \item How do you define
%         parameterized modules? \end{enumerate} \item
%         \textbf{Microarchitecture \& FPGA Design} \begin{enumerate} \item
%         What are pipeline hazards, and how can they be resolved? \item Why
%         don't modern CPUs use extremely deep pipelines (e.g., 100 stages)?
%         \item Compare cache organizations: direct-mapped, set-associative,
%         fully-associative. \item What are the key differences in design
%         approach between FPGAs and ASICs? \item Discuss trade-offs involved
%         in implementing FSMs on an FPGA. \end{enumerate} \item
%         \textbf{Miscellaneous Digital Design Topics} \begin{enumerate} \item
%         Compare clock gating and power gating. When is each used? \item How
%         does clock skew impact setup and hold timing? \item What is the
%         difference between synchronous and asynchronous resets? \item How can
%         multiplication be implemented efficiently in digital circuits? \item
%         What is the maximum possible result from multiplying two \(N\)-bit
%         numbers? \end{enumerate} \end{enumerate}

%=============================
% 10. References
%=============================
\bibliography{References}

\end{document}
